%%%% APÊNDICE A
%%
%% Texto ou documento elaborado pelo autor, a fim de complementar sua argumentação, sem prejuízo da unidade nuclear do trabalho.

%% Título e rótulo de apêndice (rótulos não devem conter caracteres especiais, acentuados ou cedilha)
\chapter{Título do Apêndice A com um Texto Muito Longo que Pode Ocupar Mais de uma Linha}
\label{cap:apendicea}

Quando houver necessidade pode-se apresentar como apêndice documento(s) auxiliar(es) e/ou complementar(es) como: legislação, estatutos, gráficos, tabelas, etc. Os apêndices são enumerados com letras maiúsculas: \autoref{cap:apendicea}, \autoref{cap:apendiceb}, etc.

Apêndices complementam o texto principal da tese com informações para leitores com especial interesse no tema, devendo ser considerados leitura opcional, ou seja, o entendimento do texto principal da tese não deve exigir a leitura atenta dos apêndices.

Apêndices usualmente contemplam provas de teoremas, deduções de fórmulas matemáticas, diagramas esquemáticos, gráficos e trechos de código. Quanto a este último, código extenso não deve fazer parte da tese, mesmo como apêndice. O ideal é disponibilizar o código na Internet para os interessados em examiná-lo ou utilizá-lo.

