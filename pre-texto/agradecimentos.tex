%%%% AGRADECIMENTOS
%%
%% Texto em que o autor faz agradecimentos dirigidos àqueles que contribuíram de maneira relevante à elaboração do trabalho.

\begin{agradecimentos}%% Ambiente agradecimentos

Certamente estes parágrafos não irão atender a todas as pessoas que fizeram parte dessa importante fase de minha vida. Portanto, desde já peço desculpas àquelas que não estão presentes entre essas palavras, mas elas podem estar certas que fazem parte do meu pensamento e de minha gratidão. 

Agradeço ao(a) meu(minha) orientador(a) Prof.(a) Dr.(a) Nome Completo, pela sabedoria com que me guiou nesta trajetória.

Aos meus colegas de sala.

A Secretaria do Curso, pela cooperação.

Gostaria de deixar registrado também, o meu reconhecimento à minha família, pois acredito que sem o apoio deles seria muito difícil vencer esse desafio. 

Enfim, a todos os que por algum motivo contribuíram para a realização desta pesquisa.


Espaço destinado aos agradecimentos (elemento opcional). Folha que contém manifestação de reconhecimento a pessoas e/ou instituições que realmente contribuíram com o(a) autor(a), devendo ser expressos de maneira simples.

Não devem ser incluídas informações que nominem empresas ou instituições não nominadas no trabalho.

Se o aluno recebeu bolsa de fomento à pesquisa, informar o nome completo da agência de fomento. Ex: Capes, CNPq, Fundação Araucária, UTFPR, etc. Incluir o número do projeto após a agência de fomento. Este item deve ser o último.

Atenção: não utilizar este exemplo na versão final. Use a sua criatividade!

\end{agradecimentos}
